\documentclass{beamer}

% Choose a theme
\usetheme{Madrid}

% Packages
\usepackage{graphicx}
\usepackage{hyperref}
\usepackage{listings}

% Title page details
\title[Writing a WSGI Web Framework]{Writing a WSGI Web Framework from Scratch}
\author{Mohammad "Kiyarash" Fazeli}
\institute{Maktabkhooneh.org}
\date{\today}

\begin{document}

% Title frame
\begin{frame}
    \titlepage
\end{frame}

% Table of Contents
\begin{frame}{Workshop Outline}
    \tableofcontents
\end{frame}

% Section 1
\section{Introduction and Historical Perspective}

\begin{frame}{Introduction and Objectives}
    \begin{itemize}[<+->]
        \item Understand the evolution of Python web application deployment.
        \item Learn the basics of WSGI and its importance.
        \item Build a simple WSGI application.
        \item Explore libraries that simplify development.
        \item Examine popular frameworks' WSGI implementations.
    \end{itemize}
\end{frame}

\begin{frame}{Historical Methods of Running Web Code: CGI}
	\textbf{CGI (Common Gateway Interface)}
            \begin{itemize}[<+->]
                \item Separate process per request.
                \item High overhead, poor scalability.
               	\item Code example
            \end{itemize}
\end{frame}

\begin{frame}{Historical Methods of Running Web Code: mod\_python}
	\textbf{mod\_python}
		\begin{itemize}[<+->]
			\item Apache module for Python.
			\item Better performance but Apache-specific.
			\item code
		\end{itemize}
		\textbf{FastCGI}
		\begin{itemize}
			\item Persistent processes.
			\item Improved performance over CGI.
		\end{itemize}
\end{frame}

\begin{frame}{Historical Methods of Running Web Code: FastCGI}
	\textbf{FastCGI}
	\begin{itemize}[<+->]
		\item Persistent processes.
		\item Improved performance over CGI.
	\end{itemize}
\end{frame}
\begin{frame}{FastCGI vs. CGI: Architectural and Performance Differences}
	\begin{itemize}[<+->]
		\item \textbf{Process Lifetime}:
		\begin{itemize}
			\item \textbf{CGI}: Separate processes for each request
			\item \textbf{FastCGI}: Long-lived, persistent processes
		\end{itemize}
		\item \textbf{Communication Mechanism}:
		\begin{itemize}
			\item \textbf{CGI}: Environment variables and I/O
			\item \textbf{FastCGI}: Efficient binary protocol
		\end{itemize}
		\item \textbf{Concurrency}:
		\begin{itemize}
			\item \textbf{CGI}: Sequential, one-at-a-time
			\item \textbf{FastCGI}: Concurrent request handling
		\end{itemize}
	\end{itemize}
\end{frame}


\begin{frame}{FastCGI vs. CGI: Architectural and Performance Differences}
	\begin{itemize}[<+->]
		\item \textbf{Performance}:
		\begin{itemize}
			\item \textbf{CGI}: Slow process creation, limited resources
			\item \textbf{FastCGI}: Faster request handling, efficient resource utilization
		\end{itemize}
		\item \textbf{Scalability}:
		\begin{itemize}
		\item Process Creation Overhead
		\item Resource Utilization
		\item Scaling Challenges
		\end{itemize}
	\end{itemize}
\end{frame}


\begin{frame}{Need for Standardization}
    \begin{itemize}
        \item Fragmentation in Python web development.
        \item Incompatibilities between servers and applications.
        \item Introduction of WSGI to provide a standard interface.
    \end{itemize}
\end{frame}

% Section 2
%\section{Case Study: Scaling Challenges}
%
%\begin{frame}{Case Study Overview}
%    \begin{itemize}
%        \item A web application facing scalability issues.
%        \item Limitations with Django and Apache.
%        \item High number of concurrent connections.
%    \end{itemize}
%\end{frame}
%
%\begin{frame}{Challenges Faced}
%    \begin{itemize}
%        \item \textbf{Django Limitations}
%            \begin{itemize}
%                \item Overhead not suitable for simple applications.
%                \item Difficult to optimize for specific needs.
%            \end{itemize}
%        \item \textbf{Apache Limitations}
%            \begin{itemize}
%                \item Process/thread per connection.
%                \item Resource-intensive under high load.
%            \end{itemize}
%    \end{itemize}
%\end{frame}
%
%\begin{frame}{Solution: Custom WSGI Framework}
%    \begin{itemize}
%        \item Built a lightweight framework tailored to the application's needs.
%        \item Improved performance and scalability.
%        \item Greater control over resource management.
%    \end{itemize}
%\end{frame}

% Section 3
\section{Introduction to WSGI}

\begin{frame}{What is WSGI?}
    \begin{itemize}
        \item \textbf{Web Server Gateway Interface}
        \item A standard interface between web servers and Python web applications.
        \item Defined in \href{https://www.python.org/dev/peps/pep-3333/}{PEP 3333}.
    \end{itemize}
\end{frame}

\begin{frame}{WSGI Components}
    \begin{itemize}
        \item \textbf{Application Callable}
        \item \textbf{\texttt{environ} Dictionary}
        \item \textbf{\texttt{start\_response} Callable}
    \end{itemize}
\end{frame}

\begin{frame}{Benefits of WSGI}
    \begin{itemize}
        \item Promotes interoperability between frameworks and servers.
        \item Simplifies deployment and scaling.
        \item Encourages the development of middleware and reusable components.
    \end{itemize}
\end{frame}

% Section 4
\section{Building a Simple WSGI Application}

\begin{frame}[fragile]{Hello World WSGI Application}
    \textbf{Code Example:}
    \begin{lstlisting}[language=Python]
def application(environ, start_response):
    status = '200 OK'
    headers = [('Content-type', 'text/plain; charset=utf-8')]
    start_response(status, headers)
    return [b"Hello, World!"]
    \end{lstlisting}
\end{frame}

\begin{frame}{Explanation of Components}
    \begin{itemize}
        \item \textbf{\texttt{environ}}: Contains request data.
        \item \textbf{\texttt{start\_response}}: Starts the HTTP response.
        \item \textbf{Return Value}: An iterable yielding the response body.
    \end{itemize}
\end{frame}

% Section 5
\section{Developing a Minimal Web Framework}

\begin{frame}{Framework Structure}
    \begin{itemize}
        \item Organize code for scalability.
        \item Separate concerns: routing, handling requests, generating responses.
    \end{itemize}
\end{frame}

\begin{frame}[fragile]{Implementing URL Routing}
    \textbf{Example Route Mapping:}
    \begin{lstlisting}[language=Python]
routes = {
    '/': home_view,
    '/about': about_view,
}
    \end{lstlisting}
    \begin{itemize}
        \item Map URLs to view functions.
        \item Handle dynamic URLs with parameters.
    \end{itemize}
\end{frame}

\begin{frame}[fragile]{Handling Requests and Responses}
    \textbf{Manual Parsing:}
    \begin{itemize}
        \item Extract query parameters from \texttt{environ}.
        \item Build response headers and body.
    \end{itemize}
\end{frame}

% Section 6
\section{Introducing WebOb and Werkzeug}

\begin{frame}{Limitations of Pure Python Implementation}
    \begin{itemize}
        \item Complexity in parsing and handling data.
        \item Potential security risks.
        \item Reinventing the wheel.
    \end{itemize}
\end{frame}

\begin{frame}[fragile]{Using WebOb}
    \textbf{Code Example:}
    \begin{lstlisting}[language=Python]
from webob import Request, Response

def application(environ, start_response):
    request = Request(environ)
    response = Response()
    response.text = "Hello, World!"
    return response(environ, start_response)
    \end{lstlisting}
\end{frame}

\begin{frame}[fragile]{Using Werkzeug}
    \textbf{Code Example:}
    \begin{lstlisting}[language=Python]
from werkzeug.wrappers import Request, Response

@Request.application
def application(request):
    return Response('Hello, World!')
    \end{lstlisting}
\end{frame}

\begin{frame}{Benefits of Using Libraries}
    \begin{itemize}
        \item Simplify request and response handling.
        \item Provide robust, tested components.
        \item Save development time and reduce errors.
    \end{itemize}
\end{frame}

% Section 7
\section{Examining Popular Frameworks}

\begin{frame}{Django's WSGI Implementation}
    \begin{itemize}
        \item Uses \texttt{wsgi.py} file.
        \item \texttt{get\_wsgi\_application()} function sets up the application.
    \end{itemize}
\end{frame}

\begin{frame}{Flask's WSGI Integration}
    \begin{itemize}
        \item The Flask app object is a WSGI application.
        \item Can access the underlying WSGI application via \texttt{app.wsgi\_app}.
    \end{itemize}
\end{frame}

\begin{frame}{Bottle's WSGI Approach}
    \begin{itemize}
        \item The default Bottle app is a WSGI application.
        \item Simple and lightweight, ideal for small applications.
    \end{itemize}
\end{frame}

% Section 8
\section{Introduction to ASGI}

\begin{frame}{What is ASGI?}
    \begin{itemize}
        \item \textbf{Asynchronous Server Gateway Interface}
        \item Designed for asynchronous Python web applications.
        \item Supports long-lived connections like WebSockets.
    \end{itemize}
\end{frame}

\begin{frame}{Why ASGI?}
    \begin{itemize}
        \item Modern web applications require asynchronous capabilities.
        \item WSGI is synchronous and cannot handle async code efficiently.
        \item ASGI enables high-performance async frameworks like FastAPI.
    \end{itemize}
\end{frame}

% Section 9
\section{Conclusion and Next Steps}

\begin{frame}{Recap}
    \begin{itemize}
        \item Explored the evolution of Python web deployment.
        \item Built a simple WSGI application and framework.
        \item Introduced libraries to simplify development.
        \item Examined popular frameworks' WSGI implementations.
        \item Briefly discussed ASGI and asynchronous programming.
    \end{itemize}
\end{frame}

\begin{frame}{Additional Resources}
    \begin{itemize}
        \item \href{https://www.python.org/dev/peps/pep-3333/}{PEP 3333: WSGI Specification}
        \item \href{https://asgi.readthedocs.io/en/latest/}{ASGI Documentation}
        \item \href{https://werkzeug.palletsprojects.com/}{Werkzeug Documentation}
        \item \href{https://docs.pylonsproject.org/projects/webob/en/stable/}{WebOb Documentation}
    \end{itemize}
\end{frame}

% Section 10
\section{Q\&A}

\begin{frame}{Questions?}
    \begin{center}
        \Large Thank you for your attention!
    \end{center}
    \begin{center}
        \normalsize Feel free to ask any questions.
    \end{center}
\end{frame}

% Closing slide
\begin{frame}{Contact Information}
    \begin{itemize}
        \item \textbf{Email}: your.email@example.com
        \item \textbf{GitHub}: \href{https://github.com/yourusername}{github.com/yourusername}
        \item \textbf{LinkedIn}: \href{https://www.linkedin.com/in/yourprofile/}{linkedin.com/in/yourprofile}
    \end{itemize}
\end{frame}

\end{document}

